\section{Logica proposicional}

\begin{questions}

\question Definir conjunto maximal consistente y probar que si $\Gamma$ es m.c. entonces: 

\begin{enumerate}[a)]
 \item $\varphi\in\Gamma$ o (exclusivo) $\neg\varphi\in\Gamma$
 \item $\varphi\in\Gamma$ sii $\Gamma\vdash\varphi$
\end{enumerate}

\tomadoEl{2016-05-09, 2015-12-21, 2014-03-07}

\begin{solution}

 {\it Def. } Un conjunto $\Gamma\subseteq$FORM es consistente si no existe $\varphi\in$FORM tal que $\Gamma\vdash\varphi$ y $\Gamma\nvdash\varphi$.
 
 {\it Def. } Un conjunto $\Gamma\subseteq$FORM es maximal consistente en SP si es consistente y para toda f\'ormula $\varphi$: o bien $\varphi\in\Gamma$, o existe $\psi$ tal que $\Gamma\cup\{\varphi\}\vdash\psi$ y $\Gamma\cup\{\varphi\}\nvdash\psi$.
  
 Para demostrar ambos puntos voy a usar una proposici\'on que dice 
 
 \begin{equation*}
  \Gamma\cup\{\neg\varphi\} \text{ es inconsistente sii } \Gamma\vdash\varphi
 \end{equation*}
 
 (Demostraci\'on en el ejercicio 7 de esta secci\'on)
 
 \begin{enumerate}[a)]
  \item No puede ser que $\varphi$ y $\neg\varphi$ est\'en en $\Gamma$ porque  ser\'ia inconsistente.
  
  Supongamos que ninguna est\'a. Como $\Gamma$ es maximal y por la proposici\'on.
  
  \begin{itemize}
   \item $\Gamma\cup\{\varphi\}$ es inconsistente $\Rightarrow \Gamma\vdash\neg\varphi$
   \item $\Gamma\cup\{\neg\varphi\}$ es inconsistente $\Rightarrow \Gamma\vdash\varphi$
  \end{itemize}

  Luego $\Gamma$ es inconsistente.
  
  \item Probar $\varphi\in\Gamma$ sii $\Gamma\vdash\varphi$. 
  
  \begin{itemize}
   \item[$\Rightarrow$)] Trivial porque o bien $\varphi$ es axioma o consecuencia inmediata de $\Gamma$ y por lo tanto $\forall v, v\vDash\varphi$. 
   \item[$\Leftarrow$)] Por la proposici\'on, $\Gamma\vdash\varphi \Rightarrow \Gamma\cup\{\neg\varphi\}$ es inconsistente. Pero $\Gamma$ es maximal consistente, luego por punto a) pasa que $\varphi\in\Gamma$.
  \end{itemize}

 \end{enumerate}

\end{solution}

\question Definir consistente, maximal consistente y demostrar el Lema de Lindenbaum. 

\tomadoEl{2014-12-11}

\question Enunciar y demostrar el Lema de Lindenbaum para la l'ogica proposicional.

\tomadoEl{2010-12-28, 2010-03-16}

\begin{solution}

{\it Lema de Lindenbaum:} Si $\Gamma \subseteq$ FORM es consistente, existe $\Delta$ maximal consistente tal que $\Gamma \subseteq \Delta$

{\it Demostraci\'on: }

Hay que encontrar el $\Delta$ tal que es m.c. y contiene a $\Gamma$. 

Enumeremos todas las f'ormulas $\varphi_1, \varphi_2, \dots$ y definimos: 

$$\Gamma_i = \left\{
  \begin{array}{l l}
  \Gamma 				& \text{si } i = 0 \\
  \Gamma_{i-1} \cup \{ \varphi_i \} 	& \text{si } \Gamma_{i-1} \cup \{ \varphi_i \} \text{ es consistente} \\ 
  \Gamma_{i-1} 				& \text{si no} 
  \end{array}
  \right.
  $$

Luego definimos $\Delta = \bigcup_{i\geq 0} \Gamma_i$. Y terminamos. $\Delta$ es \begin{inparaenum}[(a)] \item consistente, \item maximal y \item contiene a $\Gamma$ \end{inparaenum} como quer'iamos.

\begin{enumerate}[(a)]
 \item Si no fuera consistente, existir'ia $\psi$ tal que: $\Delta \vdash \psi$ y $\Delta \nvdash \psi$. En ambas derivaciones aparece \'unicamente $\{ \gamma_1, \dots, \gamma_k \} \subseteq \Delta$. Sea $j$ el m'inimo 'indice tal que $\{ \gamma_1, \dots, \gamma_k \} \subseteq \Gamma_j$. Entonces $\Gamma_j$ es inconsistente, y esto es absurdo porque se construye de $\Gamma_i$ consistentes. 
 
 \item Es maximal. Supongamos que no: existe $\varphi \notin \Delta$, tal que $\Delta\cup\{\varphi\}$ es consistente. Debe existir un $n$ tal que $\varphi_n = \varphi$. Y como $\varphi_n \notin \Gamma_n$ (porque sino estar'ia en $\Delta$), entonces $\Gamma_{n-1} \cup \{ \varphi_n \}$ es inconsistente. Luego $\Delta \cup \{ \varphi_n \}$ es inconsistente. Absurdo! 
 
 \item $\Gamma = \Gamma_0 \subseteq \Delta$.
\end{enumerate}

\end{solution}

\question Enunciar y demostrar el teorema de la Deducci\'on. 

\tomadoEl{2014-02-14}

\begin{solution}

 {\it Teorema de la Deducci\'on: } Si $\Gamma\cup\{\varphi\}\vdash\psi$ entonces $\Gamma \vdash \varphi\rightarrow\psi$.
  
 {\it Demostraci\'on: } Por inducci\'on en la demostraci\'on de $\Gamma\cup\{\varphi\}\vdash\psi$.
 
 Supongamos que $\varphi_1, \dots, \varphi_n(=\psi)$ es una derivaci\'on de $\psi$ a partir de $\Gamma\cup\{\varphi\}$.
 
 \begin{itemize}[\quad]
  \item[Caso base ($n=1$):] la derivaci\'on es una s\'ola f\'ormula ($\varphi_1=\psi$). Queremos ver que $\Gamma\vdash\varphi\rightarrow\psi$. Hay 3 posibilidades: 
  \begin{enumerate}[a)]
   \item $\psi$ es un axioma de SP. 
    \begin{enumerate}
      \item[$\Rightarrow$] $\vdash\psi$ \hfill (Por ser axioma) 
      \item[$\Rightarrow$] $\vdash\psi\rightarrow(\varphi\rightarrow\psi)$ \hfill (por SP1)
      \item[$\Rightarrow$] $\vdash\varphi\rightarrow\psi$ \hfill (por MP 1,2) 
    \end{enumerate}
    
   \item $\psi\in\Gamma$
   \begin{enumerate}
      \item[$\Rightarrow$] $\Gamma\vdash\psi$ \hfill (por $\psi\in\Gamma$)
      \item[$\Rightarrow$] $\Gamma\vdash\psi\rightarrow(\varphi\rightarrow\psi)$ \hfill (por SP1)
      \item[$\Rightarrow$] $\Gamma\vdash\varphi\rightarrow\psi$ \hfill (por MP 1,2)
    \end{enumerate}
    
   \item $\psi=\varphi$. 
    Vale porque $\vdash\varphi\rightarrow\varphi$ (visto en clase)
  \end{enumerate}

  \item[Paso inductivo:] HI: {\it ``para toda derivaci\'on $\psi'$ a partir de $\Gamma\cup\{\varphi\}$ de longitud menor a $n$ vale $\Gamma\vdash\varphi\rightarrow\psi'$''}. Queremos ver que $\Gamma\vdash\varphi\rightarrow\psi$. 
  
  Hay 4 posibilidades: 
  \begin{enumerate}[a)]
   \item $\psi$ es un axioma de SP: igual que caso base. 
   \item $\psi\in\Gamma$: igual que en caso base. 
   \item $\psi=\varphi$: igual que en caso base. 
   \item $\psi$ se infiere por MP de $\varphi_i$ y $\varphi_j$ ($i,j<n$). Sin p\'erdida de generalidad: $\varphi_j=\varphi_i\rightarrow\psi$. 
   
    Por HI {\it ``$\Gamma\cup\{\varphi\}\vdash\varphi_i$ y la derivaci\'on tiene longitud menor a $n$''} vale: $\Gamma\vdash\varphi\rightarrow\varphi_i$
    
    Por HI {\it ``$\Gamma\cup\{\varphi\}\vdash\varphi_j$ y la derivaci\'on tiene longitud menor a $n$''} vale: $\Gamma\vdash\varphi\rightarrow\varphi_j\Rightarrow$
    
    \hfill $\Gamma\vdash\varphi\rightarrow(\varphi_i\rightarrow\psi)$.
    
    Por SP2 vale: $\vdash (\varphi\rightarrow(\varphi_i\rightarrow\psi))\rightarrow((\varphi\rightarrow\varphi_i)\rightarrow(\varphi\rightarrow\psi))$.
    
    Por MP 2 veces: $\Gamma\vdash\varphi\rightarrow\psi$.
  \end{enumerate}

 \end{itemize}

\end{solution}

\question Usando (el teorema de) correctitud de la lógica proposicional probar que si (un conjunto de f'ormulas) $\Gamma$ es satisfacible entonces $\Gamma$ es consistente. 

\tomadoEl{2015-04-13, 2013-08-05, 2012-12-26}

\begin{solution}

Supongamos que $\Gamma$ es satisfacible, y sin embargo es inconsistente. 

Por lo tanto, por satisfacibilidad existe $v$ tal que $v\vDash\Gamma$.

Por inconsistencia, existe $\varphi$ tal que $\Gamma \vdash \varphi$ y $\Gamma \nvdash \varphi$.

Por correctitud de SP, $\Gamma\vdash\varphi \Rightarrow \Gamma\vDash\varphi$ y $\Gamma\nvdash\varphi\Rightarrow \Gamma\nvDash\varphi$. 

Luego $\Gamma\vDash\varphi$ y $\Gamma\nvDash\varphi$, y por \'ultimo $v \vDash \varphi$ y $v \nvDash \varphi$. Absurdo!

\end{solution}

\question Sea $\Gamma$ un conjunto de f\'ormulas de la l\'ogica proposicional. Demostrar que si $\Gamma$ es consistente entonces es satisfacible. 

\tomadoEl{2017-03-17}

\begin{solution}
 
 Dado $\Gamma$ consistente, construimos $\Gamma'\supseteq\Gamma$ m.c. (Por Lindenbaum).
 
 Definimos la interpretaci\'on $v$ tal que: 
 
 \begin{center}
  $v(p) = 1$ sii $p\in\Gamma'$
 \end{center}
 
 Demostremos que $v\vDash\varphi$ sii $\varphi\in\Gamma'$ por inducci\'on en la complejidad de $\varphi$ (es decir, cantidad de $\neg$ y $\rightarrow$ que aparecen en $\varphi$).

 \begin{enumerate}[\quad]
  \item[Caso base ($\varphi=p$):] Trivial por definici\'on de $v$. 

  \item[Paso inductivo:] HI: $v\vDash\varphi$ sii $\varphi\in\Gamma'$ para toda $\varphi$ de complejidad $<m$.
  
  Sea $\varphi$ de complejidad $m$. Hay 2 casos: 
  
  \begin{enumerate}
    \item $\varphi=\neg\psi$ tiene complejidad $m$. 
    
    \begin{itemize}
    \item[($\Rightarrow$)] $v\vDash\varphi \Rightarrow v \nvDash \psi \Rightarrow$ (x HI) $\psi\notin\Gamma' \Rightarrow \neg\psi\in\Gamma' \Rightarrow \varphi\in\Gamma'$
    \item[($\Leftarrow$)] $\varphi\in\Gamma' \Rightarrow \psi\notin\Gamma' \Rightarrow$ (x HI) $v\nvDash\psi \Rightarrow v\vDash\neg\psi \Rightarrow v\vDash\varphi$
    \end{itemize}

    \item $\varphi = \psi \rightarrow \rho$ tiene complejidad $m$. 
    
    \begin{itemize}
     \item[($\Rightarrow$)] $v\vDash\varphi \Rightarrow v\vDash (\psi \rightarrow \rho) \Rightarrow v\nvDash\psi$ o $v\vDash\rho$
     
     \begin{itemize}
      \item[-] $v\nvDash\psi \Rightarrow$ (x HI) $\psi\notin\Gamma' \Rightarrow \neg\psi\in\Gamma' \Rightarrow \Gamma'\vdash\neg\psi$
      
	Sabemos que $\vdash\neg\psi\rightarrow(\psi\rightarrow\rho)$. Por MP, $\Gamma'\vdash\psi\rightarrow\rho$. 
	
	Entonces $\psi\rightarrow\rho\in\Gamma'$.
      
      \item[-] $v\vDash\rho \Rightarrow$ (x HI) $\rho\in\Gamma' \Rightarrow \Gamma'\vdash\rho$ 
	
	Sabemos por SP1 que $\vdash\rho\rightarrow(\psi\rightarrow\rho)$. Por MP, $\Gamma'\vdash\psi\rightarrow\rho$.
	
	Entonces $\psi\rightarrow\rho\in\Gamma'$
      
     \end{itemize}
     
     \item[($\Leftarrow$)] $v\nvDash\varphi \Rightarrow v\vDash\psi$ y $v\nvDash\rho \Rightarrow$ (x HI) $\psi\in\Gamma'$ y $\rho\notin\Gamma' \Rightarrow \psi\in\Gamma'$ y $\neg\rho\in\Gamma' \Rightarrow \Gamma'\vdash\psi$ y $\Gamma'\vdash\neg\rho$.
     
     Sabemos que $\vdash\psi\rightarrow(\neg\rho\rightarrow\neg(\psi\rightarrow\rho))$. Aplicando MP 2 vences, $\Gamma'\vdash\neg(\psi\rightarrow\rho)$. 
     
     Por lo tanto, $\neg(\psi\rightarrow\rho)\in\Gamma'$. Entonces $\psi\rightarrow\rho\notin\Gamma'$.
     
    \end{itemize}

    
  \end{enumerate}


 \end{enumerate}

\end{solution}

\question Demostrar que $\Gamma \cup \{\neg\varphi\}$ es inconsistente si y s\'olo si $\varphi$ es consecuencia sint\'actica de $\Gamma$.

\tomadoEl{2014-12-22}

\begin{solution}
 
 \begin{itemize}
  \item[($\Leftarrow$)] Por propiedad: si $\Gamma\vdash\varphi \Rightarrow \Gamma\cup\{\neg\varphi\}\vdash\varphi$.
  
    Por Teorema de la Deducci\'on: $\Gamma\cup\{\neg\varphi\}\vdash\neg\varphi$.
    
    Por lo tanto, $\Gamma\cup\{\neg\varphi\}$ es inconsistente.
  \item[($\Rightarrow$)] Por hip\'otesis: existe $\psi$ tal que $\Gamma\cup\{\neg\varphi\}\vdash\psi$ y $\Gamma\cup\{\neg\varphi\}\vdash\neg\psi$. 
  
  Por Teorema de la Deducci\'on: $\Gamma\vdash\neg\varphi\rightarrow\psi$ y $\Gamma\vdash\neg\varphi\rightarrow\neg\psi$. 
  
  Se puede ver que: $\vdash(\neg\varphi\rightarrow\psi)\rightarrow((\neg\varphi\rightarrow\neg\psi)\rightarrow\varphi)$. 
  
  Por MP 2 veces: $\Gamma\vdash\varphi$.
 \end{itemize}

\end{solution}

\question Demostrar la correctitud de SP: Si $\varphi$ es teorema de la teor'ia $\Gamma$, es v'alido en toda interpretaci'on de $\Gamma$ ($\Gamma \vdash \varphi \Rightarrow \Gamma \vDash \varphi$). 

\begin{solution}

 Queremos ver que $\Gamma\vdash\varphi \Rightarrow \Gamma\vDash\varphi$. Supongamos que vale el antecedente. Es decir, existe una derivaci\'on $\varphi_1, \dots, \varphi_n$ tal que $\varphi_n=\varphi$ y (a) $\varphi_i$ es un axioma o (b) $\varphi_i\in\Gamma$ o (c) $\varphi_i$ es una consecuencia inmediata de $\varphi_k,\varphi_l, k,l<i$.
 
 Demostramos que $\Gamma\vDash\varphi$ por inducci\'on en $n$ (la longitud de la derivaci\'on):
 
 \begin{center}
  $P(n)=${\it ``si $\varphi_1, \dots, \varphi_n=\varphi$ es una derivaci\'on de $\varphi$ a partir de $\Gamma$ entonces $v\vDash\Gamma\Rightarrow v\vDash\varphi$''}
 \end{center}

 \begin{enumerate}[\quad]
  \item[Caso base ($n=1$):] Supongamos $v$ tal que $v\vDash\Gamma$. Queremos ver que $v\vDash\Gamma\Rightarrow v\vDash\varphi$. Hay 2 posibilidades: o bien $\varphi$ es axioma de SP, o pertenece a $\Gamma$. En ambos casos $v\vDash\varphi$. 

  \item[Paso inductivo:] Supongamos $v$ tal que $v\vDash\Gamma$. Supongamos que vale $P(m)$ para todo $m\leq n$. Qvq vale $P(n+1)$. Supongamos $\varphi_1, \dots, \varphi_n,\varphi_{n+1}=\varphi$ es una derivaci\'on de $\varphi$ a partir de $\Gamma$. Hay 3 posibilidades: 
  \begin{itemize}
   \item $\varphi$ es axioma de SP: igual que en caso base. 
   \item $\varphi\in\Gamma$: igual que en caso base.
   \item $\varphi$ es consecuencia inmediata de $\varphi_i$ y $\varphi_j=\varphi_i\rightarrow\varphi$ ($i,j,\leq n$). Por HI ($P(i)$ y $P(j)$), sabemos que $v\vDash\varphi_i$ y $v\vDash\varphi_i\rightarrow\varphi$. Entonces necesariamente $v\vDash\varphi$.  
  \end{itemize}

 \end{enumerate}

\end{solution}

\question Sea {\it Var} el conjunto de variables proposicionales de la l\'ogica proposicional. Probar que si $f : Var \rightarrow \{0,1\}$ es una funci\'on entonces existe una valuaci\'on $v: F \rightarrow \{0,1\}$ que extiende a $f$, donde $F$ es el conjunto de las f\'ormulas de la l\'ogica proposicional. 

\tomadoEl{2014-10-21 (por Matem\'atica)}

\begin{solution}
 
% http://www.cubawiki.com.ar/index.php?title=Final_del_21/10/14_(L%C3%B3gica_y_Computabilidad) 

Hay que probar a) la existencia y b) la unicidad de que existe una \'unica valuaci\'on $v$ que extiende a la funci\'on $f$. 

\begin{enumerate}[a)]
 \item La existencia se prueba por inducci\'on en la complejidad de la f\'ormula definiendo en cada caso c\'omo se eval\'ua. 
  
  \begin{enumerate}[\quad]
  \item[Caso base:] sea $a$ tal que $comp(a) = 0$, entonces $a$ es una variable proposicional, y por lo tanto $f(a)$ es\'a definida. Vale entonces $v(a) = f(a)$. 
  
  \item[Paso inductivo:] supongamos v\'alido $comp(a) = n$, siendo $n$ la complejidad de $a$. Veamos si $comp(a) = n+1$. 
    
    \begin{itemize}
      \item Si $a=\neg b$, entonces $comp(b) = n$. Por H.I., $v(b)$ est\'a definido. Queda que $v(a) = 1 - v(b)$. 
      
      \item Si $a=b*c$, con $* \in \{\wedge, \vee, \rightarrow \}$. Entonces $comp(b)$ y $comp(b)$ son menores a $n+1$. Por H.I. $v(b)$ y $v(c)$ est\'an definidos. Por lo tanto: 
      
      \begin{eqnarray*}
      v(a) = \min(v(b), v(c)) & \text{si } a = b \wedge c\\
      v(a) = \max(v(b), v(c)) & \text{si } a = b \vee c \\
      v(a) = \max(1-v(b), v(c)) & \text{si } a = b \rightarrow c
      \end{eqnarray*}
      
      La funci\'on $v$ queda definido para toda f\'ormula de cualquier complejidad.
    \end{itemize}
  \end{enumerate}
 \item La unicidad se prueba suponiendo que existiese otra funci\'on de valuaci\'on $w$ que extiende a $f$. 
 
 Consideremos el conjunto $I = \{a\in FORM | v(a) = w(a) \}$. 
 
 Como $w$ tambi\'en extiende a $f$, $I$ contiene a todas las variables proposicionales. Como $v$ y $w$ son ambas valuaciones, $I$ es cerrado por los conectivos por los que $FORM \subseteq I$. Es decir, $v(P) =w(P)$ para toda f\'ormula $P$. 
 
 Usando el teorema de que si un subconjunto $S$ de $A$ es cerrado por los conectivos y $S$ contiene a todas las variables proposicionales entonces $S$ contiene a todas las f\'ormulas. 
 
 (Unicidad basado en el apunte de lógica de Roberto Cignoli y Guillermo Martínez.)
\end{enumerate}

\end{solution}

\end{questions}